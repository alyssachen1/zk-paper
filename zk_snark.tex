\documentclass[12pt]{article}
\usepackage[utf8]{inputenc}
\usepackage{graphicx}
\usepackage{amsmath}
\usepackage{amsfonts}
\usepackage{amssymb}
\usepackage{subcaption}
\usepackage{float}
\usepackage{tikz}
\usepackage{listings}
\usepackage{color}
\usepackage{parskip}
\usepackage{hyperref}
\usepackage{mathrsfs}

\hypersetup{
    colorlinks=true,
    linkcolor=black,
    citecolor=black,
    filecolor=black,
    urlcolor=black
}
\usepackage[left=1in,right=1in,top=1in,bottom=1in]{geometry}
\captionsetup{justification=centering}

\title{Mathematical Construction and Optimization of Zero Knowledge Proofs (ZKPs) and Succinct Non-Interactive Arguments of Knowledge (zk-SNARKs) }
\author{Alyssa Brittany Chen}
\date{Dec. 18, 2023}

\begin{document}

\maketitle


\tableofcontents

\newpage

\section{Introduction}

\subsection{Abstract}
What defines what constitutes a mathematical proof in the simplest manner? In its essense, a proof is a rigorous way to validate a proposition is true. Similarly, Zero Knowledge Proofs (ZKPs) are a way of validating something is true, without revealing the specificities. To illustrate with a simple example, consider the case of Bob and Alice trying to determine which person is richer without revealing their individual salaries.

\section{The Construction of a Proof}
In the context of ZKPs, there exists some prover, who tries to prove the statement is true to some verifier. This protocol consists of the following properties:
\begin{enumerate}
    \item Completeness- the prover is able to convince the verifier of the statement's validity.
    \item Soundness- a malicious prover is not able to prove to a verifier a false statement.
    \item Zero-Knowledge- only the statement's validity is revealed.
\end{enumerate}

Furthermore, a polynomial satisfies the following structure:
\begin{center}
    \( a_n{}x^n + a_{n-1}{}x^{n-1} + \cdots + a_0\)
\end{center}

where each \( a_n \) term corresponds to the \(n^{th}\) term-ed coefficient. A property of polynomials states that for two any arbitrary polynomials with degree at most \(d\),
it must intersect at at most \(d\) points. 

\subsection{Computation}
For example, if we wish to prove a degree 3 polynomial with roots at \(x=1\) and \(x=2\), the Fundamental Theorem of Algebra allows us to write the polynomial
as a product of linear terms. For example:

\begin{center}
    \((x-0)(x-1)(x-2) = x^3 - 3x^2 + 2\) 
\end{center}

Therefore, the terms \((x-1)\) and \((x-2)\) are cofactors of the polynomial. In order to prove the polynomial \(p(x)\) indeed has roots at 1 and 2 without disclosing
the roots themselves, the prover must prove that \(p(x) = t(x) \cdot h(x)\), where \(t(x)\) corresponds to the target polynomial \(t(x) = (x-1)(x-2)\) and \(h(x)\) is some
arbitrary polynomial.   
    

\subsection{Properties of ZKPs}

\subsection{The Role of Elliptic Curve Pairings}

\section{Succintness \& Non-Interactivity}
\subsection{Complexities of Achieving Succinctness and Non-Interactivity}

\section{zk-SNARKs in DeFi}
\subsection{Practical Challenges and Solutions in Implementation}






\section*{References}
\begin{enumerate}
    \renewcommand{\labelenumi}{[\Alph{enumi}]}
    \item Maksym Petkus.\ \href{https://arxiv.org/pdf/1906.07221.pdf}{Why and How zk-SNARK Works: Definitive Explanation}
\end{enumerate}

\appendix
\section{Appendix}


\end{document}