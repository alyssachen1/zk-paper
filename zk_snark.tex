\documentclass[12pt]{article}
\usepackage[utf8]{inputenc}
\usepackage{graphicx}
\usepackage{amsmath}
\usepackage{amsfonts}
\usepackage{amssymb}
\usepackage{subcaption}
\usepackage{float}
\usepackage{tikz}
\usepackage{listings}
\usepackage{color}
\usepackage{parskip}
\usepackage{hyperref}
\usepackage{mathrsfs}

\hypersetup{
    colorlinks=true,
    linkcolor=black,
    citecolor=black,
    filecolor=black,
    urlcolor=black
}
\usepackage[left=1in,right=1in,top=1in,bottom=1in]{geometry}
\captionsetup{justification=centering}

\title{Mathematical Construction and Optimization of Zero Knowledge Proofs (ZKPs) and Succinct Non-Interactive Arguments of Knowledge (zk-SNARKs) }
\author{Alyssa Brittany Chen}
\date{Dec. 18, 2023}

\begin{document}

\maketitle


\tableofcontents

\newpage

\section{Introduction}

\subsection{Abstract}
What defines what constitutes a mathematical proof in the simplest manner? In its essense, a proof is a rigorous way to validate a proposition is true. Similarly, Zero Knowledge Proofs (ZKPs) are a way of validating something is true, without revealing the specificities. To illustrate with a simple example, consider the case of Bob and Alice trying to determine which person is richer without revealing their individual salaries.

\section{The Construction of a Proof}
In the context of ZKPs, there exists some prover, who tries to prove the statement is true to some verifier. This protocol consists of the following properties:
\begin{enumerate}
    \item Completeness
    \item Soundness
    \item Zero-Knowledge
\end{enumerate}
\subsection{Properties of ZKPs}
\subsection{Computation}

\section{Non-Interactivity}

\section{Succinctness}


\section*{References}
\begin{enumerate}
    \renewcommand{\labelenumi}{[\Alph{enumi}]}
    \item Maksym Petkus.\ \href{https://arxiv.org/pdf/1906.07221.pdf}{Why and How zk-SNARK Works: Definitive Explanation}
\end{enumerate}

\appendix
\section{Appendix}


\end{document}